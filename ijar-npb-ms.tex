\documentclass[authoryear, 12pt, a4paper]{elsarticle}

% ------------ packages -------------

\usepackage[utf8]{inputenc}
\usepackage[OT1]{fontenc}
\usepackage{graphicx}
\usepackage[english]{babel}

\usepackage{amsmath}
\usepackage{amsfonts}
\usepackage{amssymb}
\usepackage{amsthm}

\usepackage[usenames,dvipsnames]{xcolor}
\usepackage{todonotes}
\usepackage{etoolbox}
\usepackage{url}
%\usepackage{tikz}
%\usetikzlibrary{shapes.misc,fit}

\usepackage[bookmarks]{hyperref}

% ------------ custom defs -------------

\newcommand{\reals}{\mathbb{R}}
\newcommand{\posreals}{\reals_{>0}}
\newcommand{\posrealszero}{\reals_{\ge 0}}
\newcommand{\naturals}{\mathbb{N}}

\newcommand{\mbf}[1]{\mathbf{#1}}
\newcommand{\bs}[1]{\boldsymbol{#1}}
\renewcommand{\vec}[1]{{\bs#1}}

\newcommand{\uz}{^{(0)}} % upper zero
\newcommand{\un}{^{(n)}} % upper n
\newcommand{\ui}{^{(i)}} % upper i

\newcommand{\ul}[1]{\underline{#1}}
\newcommand{\ol}[1]{\overline{#1}}

\newcommand{\Rsys}{R_\text{sys}}
\newcommand{\lRsys}{\ul{R}_\text{sys}}
\newcommand{\uRsys}{\ol{R}_\text{sys}}

\newcommand{\Fsys}{F_\text{sys}}
\newcommand{\lFsys}{\ul{F}_\text{sys}}
\newcommand{\uFsys}{\ol{F}_\text{sys}}

\def\Tsys{T_\text{sys}}

\newcommand{\E}{\operatorname{E}}
\newcommand{\V}{\operatorname{Var}}

\newcommand{\indic}{\mathbb{I}}

\newcommand{\ber}{\operatorname{Bernoulli}} 
\newcommand{\bin}{\operatorname{Binomial}}
\newcommand{\be}{\operatorname{Beta}} 
\newcommand{\bebin}{\operatorname{Beta-Binomial}} 

\def\tmax{t_\text{max}}
\def\tnow{t_\text{now}}
\def\tpnow{t^+_\text{now}}

\newcommand{\ptk}{p^k_t}

\def\yz{y\uz}
\def\yn{y\un}
%\def\yi{y\ui}
\newcommand{\yfun}[1]{y^{({#1})}}
\newcommand{\yfunl}[1]{\ul{y}^{({#1})}}
\newcommand{\yfunu}[1]{\ol{y}^{({#1})}}

\def\yl{\ul{y}}
\def\yu{\ol{y}}
\def\nl{\ul{n}}
\def\nu{\ol{n}}
\def\nktzo{\widetilde{n}\uz_{k,t}}

\def\ykt{y_{k,t}}

\def\ykz{y\uz_k}
\def\ykn{y\un_k}

\def\yktz{y\uz_{k,t}}
\def\yktn{y\un_{k,t}}

\def\yzl{\ul{y}\uz}
\def\yzu{\ol{y}\uz}
\def\ynl{\ul{y}\un}
\def\ynu{\ol{y}\un}
\def\yil{\ul{y}\ui}
\def\yiu{\ol{y}\ui}

\def\ykzl{\ul{y}\uz_k}
\def\ykzu{\ol{y}\uz_k}
\def\yknl{\ul{y}\un_k}
\def\yknu{\ol{y}\un_k}

\def\yktzl{\ul{y}\uz_{k,t}}
\def\yktzu{\ol{y}\uz_{k,t}}
\def\yktnl{\ul{y}\un_{k,t}}
\def\yktnu{\ol{y}\un_{k,t}}

\newcommand{\ytz}[1]{y\uz_{#1,t}}
\newcommand{\ytzl}[1]{\ul{y}\uz_{#1,t}}
\newcommand{\ytzu}[1]{\ol{y}\uz_{#1,t}}

\newcommand{\ytn}[1]{y\un_{#1,t}}
\newcommand{\ytnl}[1]{\ul{y}\un_{#1,t}}
\newcommand{\ytnu}[1]{\ol{y}\un_{#1,t}}

\def\nz{n\uz}
\def\nn{n\un}
%\def\ni{n\ui}
\newcommand{\nfun}[1]{n^{({#1})}}
\newcommand{\nfunl}[1]{\ul{n}^{({#1})}}
\newcommand{\nfunu}[1]{\ol{n}^{({#1})}}

\def\nkz{n\uz_k}
\def\nkn{n\un_k}
\newcommand{\nkzfun}[1]{n\uz_{#1}}

\def\nkt{n_{k,t}}

\def\nktz{n\uz_{k,t}}
\def\nktn{n\un_{k,t}}


\def\nzl{\ul{n}\uz}
\def\nzu{\ol{n}\uz}
\def\nnl{\ul{n}\un}
\def\nnu{\ol{n}\un}
\def\nil{\ul{n}\ui}
\def\niu{\ol{n}\ui}

\def\nkzl{\ul{n}\uz_k}
\def\nkzu{\ol{n}\uz_k}
\def\nknl{\ul{n}\un_k}
\def\nknu{\ol{n}\un_k}

\def\nktzl{\ul{n}\uz_{k,t}}
\def\nktzu{\ol{n}\uz_{k,t}}
\def\nktnl{\ul{n}\un_{k,t}}
\def\nktnu{\ol{n}\un_{k,t}}

\newcommand{\ntz}[1]{n\uz_{#1,t}}
\newcommand{\ntzl}[1]{\ul{n}\uz_{#1,t}}
\newcommand{\ntzu}[1]{\ol{n}\uz_{#1,t}}

\def\taut{\tau(\vec{t})}
\def\ttau{\tilde{\tau}}
\def\ttaut{\ttau(\vec{t})}

\def\MZ{\mathcal{M}\uz}
\def\MN{\mathcal{M}\un}

\def\MkZ{\mathcal{M}\uz_k}
\def\MkN{\mathcal{M}\un_k}

\def\MktZ{\mathcal{M}\uz_{k,t}}
\def\MktN{\mathcal{M}\un_{k,t}}

\def\PZ{\Pi\uz}
\def\PN{\Pi\un}

\def\PkZ{\Pi\uz_k}
\def\PkN{\Pi\un_k}
\newcommand{\PZi}[1]{\Pi\uz_{#1}}

\def\PktZ{\Pi\uz_{k,t}}
\def\PktN{\Pi\un_{k,t}}
\newcommand{\PtZi}[1]{\Pi\uz_{#1,t}}
\newcommand{\PkZi}[1]{\Pi\uz_{k,#1}}



%\newcommand{\comments}[1]{{\small\color{gray} #1}}
\newtoggle{td}
\newcommand{\td}[1]{%
  \iftoggle{td}{%
    \todo[inline]{#1}%
  }{}%
}

% ------------ options -------------

\allowdisplaybreaks

\toggletrue{td} % show todo's
%\togglefalse{td} % hide todo's

%\biboptions{longnamesfirst,angle,semicolon}


\journal{IJAR}

\begin{document}

% ------------ frontmatter -------------

\begin{frontmatter}
\title{Bayesian Nonparametric System Reliability\\ using Sets of Priors}

\author[ein]{Gero Walter\fnref{fn1}}
\ead{g.m.walter@tue.nl}
\author[oxf]{Louis J.M. Aslett}
\ead{louis.aslett@stats.ox.ac.uk}
\author[dur]{Frank P.A. Coolen}
\ead{frank.coolen@durham.ac.uk}

\address[ein]{School of Industrial Engineering, Eindhoven University of Technology, Eindhoven, NL}
\address[oxf]{Department of Statistics, University of Oxford, Oxford, UK}
\address[dur]{Department of Mathematical Sciences, Durham University, Durham, UK}

\fntext[fn1]{Gero Walter was supported by the Dinalog project
``Coordinated Advanced Maintenance and Logistics Planning for the Process Industries'' (CAMPI).}

\begin{abstract}
Imprecise Bayesian nonparametric approach to system reliability with multiple types of components,
sets of priors through sets of canonical parameters,
leading to sets of system reliability functions,
reflection of prior-data conflict
\end{abstract}

\begin{keyword}
System reliability \sep
Survival signature \sep
Imprecise probability \sep
Bayesian Nonparametrics \sep
Prior-data conflict
\end{keyword}
\end{frontmatter}


% ------------ manuscript -------------

(author order to be discussed of course)

\section{Introduction}

(some sentences may be useful for the abstract instead)

System with components of $k=1,\ldots,K$ different types.
There are $m_k$ components of type $k$ in the system.
Components of the same type are i.i.d.\ and independent of components of other types.
Arbitrary system layout, i.e.\ any series / parallel combination, $k$ out of $n$ etc.

The survival signature allows for straightforward computation of system reliability
for systems with complex layouts and multiple component types.
It separates the (time-invariant) system structure from the time-dependent failure probabilities of components,
by partitioning the event $\Tsys > t$ according to the number of components of each type functioning at time $t$.

Based on expert assumptions for component failure distribution and component test data,
we make predictive inference for a system with components that are exchangeable
with the test components by calculating the system reliability function $\Rsys(t) = P(\Tsys > t)$.
Considering a discrete grid of time points $\{t_1, \ldots, \tmax\}$,
naturally a Bayesian model for the functioning probability at each time point arises.
To account for uncertainty in the choice of prior and due to the inherent issue with prior-data conflict (see Section~***),
we propose an imprecise or interval probability approach
based on sets of prior distributions defined through sets of canonical parameters.
This approach also allows to adequately model weak or partial prior information.
Especially prior near ignorance can be modelled much more meaningfully than through usual so-called noninformative priors. 

Our method produces a set of discrete system reliability functions,
or, for each time point $t$, an interval for the system survival probability,
that appropriately reflects uncertainty due to vague prior information, the amount of test data, and prior-data conflict.

***main contribution of the paper:
set of priors thing,
guidelines on how to choose the parameters;
this is easier in terms of $(\nz, \yz)$ than in terms of $(\alpha, \beta)$***

This paper implements the nonparametric approach described in Section~4 of \citet{2015:bayessurvsign}
and extends it to sets of priors.

The paper is organised as follows.
In Section~\ref{sec:survsign}, we describe ***


\section{Survival Signature}
\label{sec:survsign}

In mathematical theory of reliability, the main focus is on the functioning of a system given the functioning, or not, 
of its components and the structure of the system. The mathematical concept which is central to this theory is the 
\emph{structure function} \citep{BP75}. For a system with $m$ components, let state vector 
$\underline{x} = (x_1,x_2,\ldots,x_m) \in \{0,1\}^m$, with $x_i=1$ if the $i$th component functions 
and $x_i=0$ if not. The labelling of the components is arbitrary but must be fixed to define $\underline{x}$. 
The structure function $\phi : \{0,1\}^m \rightarrow \{0,1\}$, defined for all possible $\underline{x}$, takes 
the value 1 if the system functions and 0 if the system does not function for state vector $\underline{x}$. 
Most practical systems are coherent, which means that $\phi(\underline{x})$ 
is non-decreasing in any of the components of $\underline{x}$, so system functioning cannot be improved by worse performance 
of one or more of its components. The assumption of coherent systems is also convenient from the perspective of uncertainty
quantification for system reliability. It is further logical to assume that $\phi(\underline{0})=0$ and $\phi(\underline{1})=1$, 
so the system fails if all its components fail and it functions if all its components function. 

For larger systems, working with the full structure function may be complicated, and one may particularly
only need a summary of the structure function in case the system has exchangeable components of one or more
types. We use the term `exchangeable components' to indicate that the failure times of the components in the system
are exchangeable \citep{DF74}. \citet{2012:survsign} introduced such a summary,
called the \emph{survival signature}, 
to facilitate reliability analyses for systems with multiple types of components. In case of just a single type of components, 
the survival signature is closely related to the system signature \citep{Sa07}, which is well-established and the topic of many
research papers during the last decade. However, generalization of the signature to systems with
multiple types of components is extremely complicated (as it involves ordering order statistics of different
distributions), so much so that it cannot be applied to most practical systems. In addition to the 
possible use for such systems, where the benefit only occurs if there are multiple components of the 
same types, the survival signature is arguably also easier to interpret than the signature. 

Consider a system with $K\ge 1$ types of components, with $m_k$ components of type $k \in \{1,\ldots,K\}$ and 
$\sum_{k=1}^K m_k = m$. Assume that the random failure times of components of the same type are exchangeable \citep{DF74}.
Due to the arbitrary ordering of the components in the state vector, components of the same type can be grouped together, 
leading to a state vector that can be written as 
$\underline{x} = (\underline{x}^1,\underline{x}^2,\ldots,\underline{x}^K)$, with 
$\underline{x}^k = (x^k_1,x^k_2,\ldots,x^k_{m_k})$ the sub-vector representing the states of the components of type $k$. 

The \emph{survival signature} for such a system, denoted by $\Phi(l_1,\ldots,l_K)$, with $l_k=0,1,\ldots,m_k$ 
for $k=1,\ldots,K$, is defined as the probability for the event that the system functions given that \emph{precisely} $l_k$ of its 
$m_k$ components of type $k$ function, for each $k\in \{1,\ldots,K\}$ \citep{2012:survsign}.
Essentially, this creates a $K$-dimensional partition for the event $\Tsys > t$, such that $\Rsys(t) = P(\Tsys > t)$
can be calculated using the law of total probability:
\begin{align}
\label{eq:rsyswithsurvsign}
P(\Tsys > t)
 &= \sum_{l_1=0}^{m_1} \cdots \sum_{l_K=0}^{m_K} P(\Tsys > t \mid C^1_t = l_1,\ldots, C^K_t = l_K) \nonumber\\
 &  \hspace*{24ex}                        \times P\Big( \bigcap_{k=1}^K \{ C^k_t = l_k\} \Big) \nonumber\\
 &= \sum_{l_1=0}^{m_1} \cdots \sum_{l_K=0}^{m_K} \Phi(l_1, \ldots, l_K)
                                                 P\Big( \bigcap_{k=1}^K \{ C^k_t = l_k\} \Big) \,,
%                                                 \prod_{k=1}^K P(C^k_t = l_k) \,.
\end{align}
where $C^k_t \in \{0, 1, \ldots, m_k\}$ denotes
the random number of components of type $k$ functioning at time $t$. 

For calculating the survival signature based on the structure function, observe that
there are $\binom{m_k}{l_k}$ state vectors $\underline{x}^k$ with $\sum_{i=1}^{m_k} x^k_i = l_k$. Let $S^k_{l_k}$ 
denote the set of these state vectors for components of type $k$ and let $S_{l_1,\ldots,l_K}$ denote the set of 
all state vectors for the whole system for which $\sum_{i=1}^{m_k} x^k_i = l_k$, $k=1,\ldots,K$. Due to the 
exchangeability assumption for the failure times of the $m_k$ components of type $k$, all the state vectors 
$\underline{x}^k \in S^k_{l_k}$ are equally likely to occur, hence \citep{2012:survsign}
\begin{align}
\label{eq:surv-sig}
\Phi(l_1,\ldots,l_K)
 &= \left[ \prod_{k=1}^K \binom{m_k}{l_k}^{-1} \right] \times \sum_{\underline{x} \in S_{l_1,\ldots,l_K}} \phi(\underline{x})\,.
\end{align}
%
%Let $C^k_t \in \{0,1,\ldots,m_k\}$ denote the number of components of type $k$ in the system that function at time $t>0$.
%Then, for system failure time $\Tsys$,
%\begin{align*}
%P(\Tsys > t) &= \Rsys(t) = \sum_{l_1=0}^{m_1} \cdots \sum_{l_K=0}^{m_K}  \Phi(l_1,\ldots,l_K) P\Big(\bigcap_{k=1}^K \{C^k_t = l_k\}\Big)\,.
%\end{align*}
It should be emphasized that when using the survival signature,
there are no restrictions on dependence of the failure times of components of different types,
as the probability $P(\bigcap_{k=1}^K \{C^k_t = l_k\})$ can take any form of dependence into account,
for example one can include common-cause failures quite straightforwardly into this approach \citep[see][]{CCM15}.
However, there is a substantial simplification
if one can assume that the failure times of components of different types are independent,
and even more so if one can assume that the failure times of components of type $k$ 
are conditionally independent and identically distributed with CDF $F_k(t)$.
With these assumptions, we get
\begin{align*}
\Rsys(t) &= \sum_{l_1=0}^{m_1} \cdots \sum_{l_K=0}^{m_K} \left[ \Phi(l_1,\ldots,l_K)
            \prod_{k=1}^K \left( \binom{m_k}{l_k} [F_k(t)]^{m_k-l_k} [1-F_k(t)]^{l_k} \right) \right]\,.
\end{align*}
We will employ both assumptions in this paper,
leading to $C^k_t$ having a Beta-Binomial distribution,
giving us a closed form expression for $P(C^k_t = l_k)$ for all $t$, $k$, and $l_k$.

The main advantage of the survival signature, in line with this property of the signature for systems with a single type of 
components \citep{Sa07}, is that the information about the system structure is fully 
separated from the information about functioning of the components, which simplifies related statistical inference as well as
considerations of optimal system design. In particular for study of system reliability over time, with the structure of the system, 
and hence the survival signature, not changing, this separation also enables relatively straightforward statistical inferences. 

There are several relatively straightforward generalizations of the use of the survival signature.
The probabilities for the numbers of functioning components can be generalized to lower and upper probabilities,
as e.g.\ done by \citet{CCMA14} within the nonparametric predictive inference (NPI) framework of statistics \citep{Co11},
where lower and upper probabilities for the events $C_k = l_k$
are inferred from test data on components of the same types as those in the system.
This is an approach that is also followed in the current paper, but with the use of generalized Bayesian inference instead of NPI.
***This step is less trivial here as we must ensure to have probability distributions for these events,
thus summing to one over $l_k=0,1,\ldots,m_k$ for each type $k$.
***covered because of parametric distribution, min and max over parameter sets (see section 6)***
***For coherent systems this is not very complicated due to the monotonicity of the survival signature,
see \citet{CCMA14} where the method for dealing with imprecision on the random quantities $C_k$ is presented.***
\td{Leave out the sentences framed with `***'?}
Recently, \citet{CCM16} suggested the generalization of the structure function from a binary function to a probability,
to reflect uncertainty about system functioning for known states of its components.
A further suggestion was to generalize this to lower and upper structure functions,
within the theory of imprecise probability \citep{itip,CU11}.
These suggestions are also easy to incorporate into system reliability with the use of the survival signature,
which in the latter case also becomes an imprecise probability.  

The survival signature is a powerful but rather basic concept.
As such, further generalizations are conceptually easy,
for example one can straightforwardly generalize the survival signature to multi-state systems
such that it again summarizes the structure function
in a manner that is sufficient for a range of uncertainty quantifications for the system reliability. 


\section{Nonparametric Bayesian Approach for Component Reliability}
\label{sec:nonparamapproach}

Let us denote the random failure time of component number $i$ of type $k$ by $T^k_i$, $i = 1, \ldots m_k$.
The failure time distribution can be written in terms of the cdf $F^k(t) = P(T^k_i \le t)$,
or in terms of the reliability function $R^k(t) = P(T^k > t) = 1 - F^k(t)$,
also known as the survival function.
For a nonparametric description of $R^k(t)$,
we consider a set of time points $t$, $t \in {\cal T} = \{t_1, \ldots, \tmax\}$.

For each time point $t$, the functioning of a single component of type $k$ at time $t$ (functioning: 1, failed: 0)
is Bernoulli distributed with a suitable functioning probability (as, $1-$failure probability) $\ptk$, so
\begin{align*}
P\big(\indic(T^k_i > t) = 1\big) &= \ptk\,, \\
P\big(\indic(T^k_i > t) = 0\big) &= 1 - \ptk\,,
\end{align*}
in short, $\indic(T^k_i > t) \sim \ber(\ptk)$, $i = 1, \ldots, m_k$, $t \in {\cal T}$.

The set of functioning probabilities $\{ \ptk, t \in {\cal T}\}$
defines a discrete failure time distribution for components of type $k$ through
\begin{align*}
R^k(t_j) &= P(T^k > t_j) = p^k_{t_j},\ t_j = t_1, \ldots, \tmax\,.
\end{align*}
We can express this failure time distribution also through the probability mass function (pmf) and the discrete hazard function,
\begin{align*}
f^k(t_j) &= P\big(T^k \in (t_j,t_{j+1}]\big) = p^k_{t_j} - p^k_{t_{j+1}}\,,\\ 
h^k(t_j) &= P\big(T^k \in (t_j,t_{j+1}]\mid T^k > t_j\big) = \frac{f^k(t_j)}{R^k(t_j)}\,. % or R^k(t_{j-1}) ???
\end{align*}
We assume the time grid $\cal T$ is dense enough for the application at hand,
otherwise it would be possible to assume, e.g.,
$R^k(t) = p^k_{t_j}, t \in [t_j, t_{j+1})$,
or taking $p^k_{t_j}$ and $p^k_{t_{j+1}}$ as upper and lower bounds for $R^k(t)$, $t \in [t_j, t_{j+1})$.

\td{***illustrate discrete distribution with a graph of R, f, h?***}

Due to the independence assumption for components of the same type, 
the number of functioning components out of the $m_k$ components of type $k$
is binomially distributed, $C^k_t = \sum_{i=1}^{m_k} \indic(T^k_i > t) \sim \bin(\ptk, m_k)$.
%(maybe not relevant for this manuscript, but would it make sense to take Kaplan-Meier estimates for the $p^k_t$'s,
%or the NPI bounds for Kaplan-Meier?)

The $\ptk$'s can be taken such that they reflect, e.g., a bathtub curve for the corresponding hazard rate $h^k(t_j)$.
Naturally, $p^k_{t_j} \ge p^k_{t_{j+1}}$ should hold (assuming no repair).

However, directly choosing the $\ptk$'s is hard,
and fixing values for all $p^k_t$ would furthermore neglect any inherent uncertainty in the particular choice.  
To account for this uncertainty, one can express knowledge on $\ptk$ through a prior distribution.
A convenient and natural choice is $\ptk \sim \be(\alpha^k_t, \beta^k_t)$, a conjugate prior,
leading to the posterior being again Beta.

Having observed the lifetime data $\vec{t}^k = (t^k_1, \ldots, t^k_{n_k})$,
at any fixed time $t \in {\cal T}$ this results in the observation from the Binomial model described above,
$s^k_t = \sum_{i=1}^{n_k} \indic(t^k_i > t)$.
The posterior is then $\ptk \mid s^k_t \sim \be(\alpha^k_t + s^k_t, \beta^k_t + n_k - s^k_t)$.
The combination of a Binomial observation model with a Beta prior is often called Beta-Binomial model.

The posterior predictive distribution for the number of components surviving at time $t$
of the $m_k$ components in the system, based on the lifetime data and the prior information,
is then a so-called Beta-Binomial distribution,
$C^k_t \mid s^k_t \sim \bebin(m_k, \alpha^k_t + s^k_t, \beta^k_t + n_k - s^k_t)$.
That is, we have
\begin{align*}
P(C^k_t = l_k \mid s^k_t) &= {m_k \choose l_k} \frac{B(l_k + \alpha^k_t + s^k_t, m_k - l_k + \beta^k_t + n_k - s^k_t)}
                                                    {B(\alpha^k_t + s^k_t, \beta^k_t + n_k - s^k_t)} \,,
\end{align*}
where $B(\cdot, \cdot)$ is the Beta function.
This posterior predictive distribution is basically a Binomial distribution
where the functioning probability $\ptk$ takes values in $[0,1]$
with the probability given by the posterior on $\ptk$.

To wrap up, to calculate the system reliability according to \eqref{eq:rsyswithsurvsign},
for each component type $k$,
we need lifetime data $\vec{t}^k$,
and have to choose $2 \times |{\cal T}|$ parameters
to specify the prior distribution for the discrete survival function of $T^k_i$.
In total, values for $2 \times |{\cal T}| \times K$ parameters must be chosen.


\section{Reparametrisation of the Beta Distribution}

The parametrisation of the Beta distribution used above is common,
and allows to interpret $\alpha^k_t$ and $\beta^k_t$ as
hypothetical numbers of functioning and failed components of type $k$ at time $t$, respectively.
However, for our generalisation to sets of priors,
it is useful to consider a different parametrisation.

For discussing this reparametrisation, we will drop the super- and subscript $k$ and $t$ for a while.
Instead of $\alpha$ and $\beta$, we consider the parameters $\nz$ and $\yz$, where
\begin{align}
\nz &= \alpha + \beta &
&\text{and} &
\yz &= \frac{\alpha}{\alpha + \beta} \,,
\label{eq:reparam}
\end{align}
or vice versa, $\alpha = \nz\yz$ and $\beta = \nz(1-\yz)$.
The upper index ${}\uz$ is used to identify $\nz$ and $\yz$ as prior parameter values,
in contrast to their posterior values $\nn$ and $\yn$
obtained after observing $n$ failure times, see below.
$\nz$ and $\yz$ are sometimes called \emph{canonical} parameters,
identified from rewriting the density in a canonical form;
see, e.g., \citet[pp.~202 and 272f]{2000:bernardosmith}.
This canonical form gives a joint structure to all conjugate priors in exponential families.

From the properties of the Beta distribution,
it follows that $\yz = \E[p]$ is the prior expectation for the functioning probability $p$,
and that the higher $\nz$, the more probability weight will be concentrated around $\yz$,
as $\V(p) = \frac{\yz (1-\yz)}{\nz + 1}$.
Furthermore, $\nz$ can be interpreted as a pseudocount or prior strength,
this is clear from the interpretation of $\alpha$ and $\beta$ and \eqref{eq:reparam},
and will be further illustrated below through revisiting the posterior
%based on observed lifetimes $\vec{t} = (t_1, \ldots, t_{n})$.
given $s$ of $n$ components functioning.
We have that
$p \mid s$ is Beta distributed with the updated parameters
\begin{align}
\nn &= \nz + n\,, &
\yn &= \frac{\nz}{\nz + n} \cdot \yz + \frac{n}{\nz + n} \cdot \frac{s}{n}\,.
\label{eq:nyupdate}
\end{align}
After seeing that $s$ out of $n$ components function (at time $t$),
the posterior mean $\yn$ for $p$ is a weighted average of
the prior mean $\yz$ and $s/n$ (the fraction of functioning components in the data),
with the weights $\nz$ and $n$, respectively.
We see that $\nz$ plays the same role for the prior mean $\yz$
as the sample size $n$ for the observed mean $s/n$,
thus the notion of pseudocount.
Indeed, the higher $\nz$, the higher the weight for $\yz$
in the weighted average calculation of $\yn$,
so $\nz$ gives the strength of the prior as compared to the sample size $n$.
%The parameters $\nz$ and $\yz$ have a more intuitive interprtation,

The Beta-Binomial probability mass function (pmf) in terms of the updated parameters $\nn$ and $\yn$,
now again with indicators $t$ and $k$,
is therefore
\begin{align}
P(C^k_t = l_k \mid s^k_t) &= {m_k \choose l_k} \frac{B(l_k + \nn_{k,t}\yn_{k,t}, m_k - l_k + \nn_{k,t}(1-\yn_{k,t}))}
                                                    {B(\nn_{k,t}\yn_{k,t}, \nn_{k,t}(1-\yn_{k,t}))} \,,
\label{eq:postpredCny}
\end{align}
and the corresponding cumulative mass function (cmf) is given by
\begin{align}
F_{C^k_t\mid s^k_t}(l_k) &= P(C^k_t \le l_k \mid s^k_t) = \sum_{j_k=0}^{l_k} P(C^k_t = j_k \mid s^k_t)\,.
\label{eq:postpredCnycmf}
\end{align}


The parametrization in terms of prior mean and prior strength
enables us to see that in this conjugate setting,
learning from data amounts to averaging between prior and data.
While this allows for the tractability of the model,
it also comes with a serious drawback:
When observed data are very much different from what is assumed in the prior,
this conflict is simply averaged out,
and is not reflected in the posterior or the posterior predictive.

As a simple example, assume that we expect $\ptk$ to be about $0.75$ for a certain $k$ and $t$,
so we choose $\yktz = 0.75$,
and that we value this choice of mean functioning probability with $\nktz = 8$,
i.e., like having seen $8$ observations with a mean $0.75$.
Assume further that there are $n^k = 16$ components of type $k$ in the test data.
When we observe $s^k_t = 12$ components functioning at time $t$,
so $s^k_t/n^k = 0.75$ as we expect,
then the updated parameters are $\nktn = 24, \yktn=0.75$.
Unexpectedly observing that no component functions at time $t$ instead
leads to parameters $\nktn = 24, \yktn=0.25$.
The prior and the posteriors based on these two observation scenarios
are depicted in Figure~\ref{fig:singleprior-pdc}
through their Beta densities and cdfs (left)
and corresponding Beta-binomial probabilities $P(C^k_t = l_k\mid s^k_t)$ and cmfs
for the case of $m_k = 5$ (right).

Due to symmetry, we see that both posteriors have the same variance or spread,
although arising from two fundamentally different scenarios.
Posterior 1 is based on data exactly according to prior expectations;
the increase in confidence on $\ptk \approx 0.75$
is reflected in a more pointed posterior density,
and the posterior predictive is changed only slightly.
It is quite disturbing to see the same degree of confidence in Posterior 2,
being based on data in sharp conflict with prior expectations.
Posterior 2 places most probability weight around $0.25$,
averaging between prior expectation and data,
with the same variance or spread as Posterior 1.
Instead of acknowledging the conflict between observed functioning probability zero
and expected functioning probability $0.75$ by increased variance,
Posterior 2 gives a false sense of certainty,
also reflected in the posterior predictive,
where both Posterior 1 and 2 have the same variance as well.

\begin{figure}
\includegraphics[width=\textwidth]{singleprior-pdc2}
\caption{Beta densities (top left) and cdfs (bottom left),
with the corresponding Beta-binomial probability mass functions (top right) and cumulative mass functions (bottom right),
for a prior with $\nktz = 8, \yktz = 0.75$,
and posteriors based on $n^k_t=16$ observations with $s^k_t=12$ (Posterior~1) and $s^k_t=0$ (Posterior~2), respectively.
Data for Posterior~1 confirm prior assumptions,
while data for Posterior~2 are in conflict with the prior.
However, this conflict is averaged out,
and Posterior 1 and Posterior 2 have the same spread, both in the posterior pdf/cdf and the posterior predictive pmf/cmf,
such that Posterior 2 gives a false sense of certainty despite the massive conflict between prior and data.}
\label{fig:singleprior-pdc}
\end{figure}

To counter this unwanted behaviour,
we propose to use an imprecise probability approach
based on sets of Beta priors, described next.


\section{Sets of Beta Priors}

As was shown by \citet{2009:WalterAugustin}, %Walter \& Augustin (2009),
we can have both tractability and meaningful reaction to prior-data conflict
by using sets of priors $\MktZ$ produced by parameter sets $\PktZ = [\nktzl, \nktzu] \times [\yktzl, \yktzu]$. 
(A detailed discussion of different choices for $\PktZ$ is given in \citet[\S 3.1]{2013:diss-gw}.)
In our model, each prior parameter pair $(\nktz, \yktz) \in \PktZ$
corresponds to a Beta prior, thus $\MktZ$ is a set of Beta priors.
The set of posteriors $\MktN$ is obtained by updating each prior in $\MktZ$ according to Bayes' Rule.
This element-by-element updating may seem ad-hoc, but can be rigorously justified
as ensuring coherence \citep[\S ***]{1991:walley}, and was termed ``Generalized Bayes' Rule'' by \citet[\S 6***]{1991:walley}.
Due to conjugacy, $\MktN$ is a set of Beta distributions with parameters $(\nktn, \yktn)$
obtained from updating $(\nktz, \yktz) \in \PktZ$ according to \eqref{eq:nyupdate},
and so can be described efficiently through the set of updated parameters
\begin{align*}
\PktN &= \Big\{ (\nktn, \yktn) \mid (\nktz, \yktz) \in \PktZ = [\nktzl, \nktzu] \times [\yktzl, \yktzu] \Big\}\,.
\end{align*}
Such rectangular prior parameter sets $\PktZ$ have been shown
to balance desirable model properties and ease of elicitation quite well.
%
For each component type $k$ and time point $t$,
one has to choose only the four parameters $\nktzl, \nktzu, \yktzl, \yktzu$
(so $4 \times |{\cal T}|$ parameters are needed to define the set of prior distributions characterizing
prior knowledge on the survival function of components of type $k$);
notwithstanding this, the model has favourable inference properties.

Central to these inference properties is that
while the prior parameter set $\PktZ$ is rectangular,
the posterior parameter set $\PktN$ is not rectangular anymore;
indeed, the shape of $\PktN$ depends on the presence of prior-data conflict,
which is naturally operationalized as $s^k_t/n^k_t \not\in [\yktzl, \yktzu]$:
prior-data conflict is present when the observed fraction of functioning components
is outside its expected range.

In absence of prior-data conflict, 
$\PktN$ shrinks in the $\ykt$ dimension;
how much it shrinks depends on $\nktz \in [\nktzl, \nktzu]$,
leading to the so-called spotlight shape depicted in Figure***.
As $\yktn$ gives the posterior expectation for the functioning probability $p_t^k$,
shorter $\yktn$ intervals mean more precise knowledge about $p_t^k$.
Also, the variance interval for $p_t^k$ will decrease
as the Beta distributions in $\MktN$ will be more pointed
due to the increase of $\nktz$ to $\nktn$.

In case of prior-data conflict ($s^k_t/n^k_t \not\in [\yktzl, \yktzu]$),
$\PktN$ has instead the so-called banana shape,
arising from $\yktn$ intervals being shifted closer towards $s^k_t/n^k_t$
for lower $\nktn$ values than for higher $\nktn$ values, see Figure***.
Overall, this results in a wider $\yktn$ interval as compared to the non-conflict case, 
reflecting the extra uncertainty due to prior-data conflict,
thus making more cautious probability statements about $p_t^k$.

\td{Figure with $\PN$s and sets of beta cdfs for conflict and no conflict case.}

Furthermore, with sets of Beta priors it is also possible
to express prior ignorance in a more adequate way,
by letting $\yktzl \to 0$ and $\yktzu \to 1$
for some or all $t \in {\cal T}$.
(It is not advisable to choose $\yktzl = 0$ and $\yktzu = 1$,
as this can lead to improper posterior predictive distributions.
E.g., for any $t < \min(\vec{t}^k)$,
we would get $\yktnu = 1$, leading to one argument of the Beta function
in the denominator of \eqref{eq:postpredCny} being zero.)
Choosing these limits for $\yktz$ means that we are prepared to give trivial bounds for the functioning probability only,
and do not wish to make any commitments about $p^k_t$ a priori.
Compare this to the uniform prior over $[0,1]$, a usual choice of `noninformative' prior,
being a Beta prior with $\alpha^k_t = \beta^k_t = 1$, i.e., $\nktz = 2$, $\yktz = 0.5$.
Taking this for all $t \in {\cal T}$ means that we expect the component reliability function
to be on average 1/2 for all $t$, a very peculiar assumption,
and, in our view, quite incompatible with the notion of prior ignorance.

For an informative prior the choice of $\nktzl$ is relevant for posterior inferences in case of prior-data conflict,
as then one of the bounds for $\ykt$ is attained for $\nktzl$.
In a near-noninformative setting, the choice of $\nktzl$ is instead not relevant,
as both $\yktnl$ and $\yktnu$ are obtained with $\nktzu$;
indeed $\yktzl > 0$ and $\yktzu < 1$ can be chosen such that
$\frac{s^k_t}{n_k} \in [\yktzl, \yktzu]$ for all $t \in \big(\min(\vec{t}^k), \max(\vec{t}^k)\big)$.
Obviously, one cannot have prior-data conflict in case of prior near ignorance.

\td{***called near-noninformative, for near-noninformative sets when $\yz$ is not bounded see
benavoli zaffalon papers***}


\section{Sets of System Reliability Functions}

To obtain the lower and upper bound for the system reliability function $\Rsys(t)$,
we now need to minimise and maximise Equation~\eqref{eq:rsyswithsurvsign} over $\PtZi{1}, \ldots, \PtZi{K}$ for each $t$,
where the posterior predictive probabilities for $C^k_t$ are given by the Beta-Binomial pmf \eqref{eq:postpredCny}.
It turns out that we need to optimise over $\ntz{1}, \ldots, \ntz{K}$ only, %$\nktz$ ($k=1,\ldots,K$) only,
as lower and upper bounds for $\Rsys(t)$ are obtained for $\yktzl$ and $\yktzu$, respectively.

To see this,
remember that $\yktz$ gives the prior expected functioning probability for a component of type $k$ at time $t$.
From Equation~\eqref{eq:nyupdate} (right) we see that $\yktn$ is increasing in $\yktz$,
%the posterior parameter set bounds are the transformed prior parameter set bounds,
%i.e., $\yktzu$ corresponds to the upper bound of $\PktN$, for all $\nktz \in \PktZ$,
such that for each fixed $\nktz$, $\yktzu$ will maximise the posterior expected functioning probability $\yktn$.
%
Now, a higher posterior expected functioning probability means
higher probability for many components out of $m_k$ surviving,
i.e., probabilities for the larger $l_k$'s are higher, and in turn for the lower $l_k$'s are lower.
Indeed, the Beta-Binomial distribution is stochastically ordered for varying $\yktn$, i.e.,
its cmf \eqref{eq:postpredCnycmf} is decreasing in $\yktn$ for all $l_k$,
such that the cmfs for different values of $\yktn$ do not cross each other.
This means that for each fixed $\nktz$, the Beta-Binomial cmf \eqref{eq:postpredCnycmf} is minimized for $\yktzu$ in all $l_k$,
such that the probability weights given to high values of $l_k$ are maximised. %for $\yktzu$.
For coherent systems, $\Phi$ is non-decreasing in each of its arguments $l_1,\ldots,l_K$,
thus $\Rsys(t)$ is maximised by putting most probability weight on the highest $l_k$ values for each $k$,
which is in turn obtained for $\yktzu$, $k=1,\ldots,K$.
We therefore have
%
%for a certain $k$, a high probability on high $l_k$'s is never worse than a high probability on lower $l_k$'s,
%keeping all other component type expressions fixed.
%***Monotonicity in $\yz_{k,t}$, so we get, for each $t \in {\cal T}$,
\begin{align*}
\lefteqn{\lRsys(t \mid \vec{t}^1, \ldots, \vec{t}^K)}\\
 &= \min_{\PtZi{1}, \ldots, \PtZi{K}} \Rsys(t \mid \PtZi{1}, \ldots, \PtZi{K}, \vec{t}^1, \ldots, \vec{t}^K) \\
 &= \min_{\nz_{1,t},\ldots,\nz_{K,t}} 
    \sum_{l_1=0}^{m_1} \cdots \sum_{l_K=0}^{m_K} \Phi(l_1, \ldots, l_K)
                                                 \prod_{k=1}^K P(C^k_t = l_k \mid \yktzl, \nktz, s^k_t) \\
 &= \min_{\nz_{1,t},\ldots,\nz_{K,t}} 
    \sum_{l_1=0}^{m_1} \cdots \sum_{l_K=0}^{m_K} \Phi(l_1, \ldots, l_K) \times \\ & \hspace*{12ex}
    \prod_{k=1}^K {m_k \choose l_k} \frac{B(l_k + \nn_{k,t}\ynl_{k,t}, m_k - l_k + \nn_{k,t}(1-\ynl_{k,t}))}
                                         {B(\nn_{k,t}\ynl_{k,t}, \nn_{k,t}(1-\ynl_{k,t}))} \,.
\end{align*}
In the following, we will give some guidelines on how to choose the parameter sets $\PkZi{1}, \ldots, \PkZi{\tmax}$
which define the set of prior discrete reliability functions for components of type $k$.
We think this is much easier in terms of $\nz$ and $\yz$ than it would be in terms of $\alpha$ and $\beta$.
%***How to choose the sets of priors for the survival function of type $k$ components,
%i.e. how to choose $\PkZi{1}, \ldots \PkZi{\tmax}$?***

As mentioned in Section~\ref{sec:nonparamapproach}, for the actual functioning probabilities $\ptk$
it should hold that $p^k_{t_j} \ge p^k_{t_{j+1}}$. ***or $p^k_{t_i} \ge p^k_{t_i}$ for all $i < j$.
%***not necessary to ensure $p^k_{t_j} \ge p^k_{t_{j+1}}$ via $\yzl_{k,t_j} \ge \yzu_{k,t_{j+1}}$
%as for relatively dense $t$ grids, neighbouring intervals should be able to overlap***
%(Prior guess for $p^k_t$ interpretation would say that $\yktz$ cannot increase,
%but does this need to hold for the prior guess range bound $\yktzu$?)***
This naturally translates to conditions on the prior guess for $\ptk$,
such that we propose that 
$\yzu_{k,t_j} \ge \yzu_{k,t_{j+1}}$ and $\yzl_{k,t_j} \ge \yzl_{k,t_{j+1}}$ should hold.
Due to $s^k_t/n_k$ decrasing in $t$ and the weighted average property of the update step for $\ykt$,
this ensures that 
$\ynu_{k,t_j} \ge \ynu_{k,t_{j+1}}$ and $\ynl_{k,t_j} \ge \ynl_{k,t_{j+1}}$.
In a situation where we are quite sure about the functioning probability for low $t$,
but are unsure about what happens for larger $t$ and so want to be maximally cautious for those $t$,
then we can let $\yktzl$ drop to (almost) 0, but $\yktzu$ should not increase.
%***Require $\yzu_{k,t_j} \ge \yzu_{k,t_{j+1}}$ and $\yzl_{k,t_j} \ge \yzl_{k,t_{j+1}}$.
%If we know, e.g., pretty much what's going on for low $t$
%but want to be much more cautious for high $t$,
%then $\yktzl$ can drop to 0, but $\yktzu$ cannot increase.

We think it is not advisable to express certainty about expected functioning probabilities
with $\nktz$ bounds that vary strongly over $t$.
%***Better to express certainty of prior guess via $\nktz$ bounds?\\
With (strongly) differing $\nktz$ bounds, monotonicty for $\yktn$ bounds cannot be ensured.
%it can happen that, e.g., $\ynu_{k,t_j} < \ynu_{k,t_{j+1}}$!\\
E.g., when $\yzu_{k,t_j} = \yzu_{k,t_{j+1}}$, $\yzl_{k,t_j} = \yzl_{k,t_{j+1}}$,
and $s^k_{t_j}/n_k \in [\yzl_{k,t_j}, \yzu_{k,t_j}]$ (so there is no prior-data conflict),
and furthermore no failures happening in $[t_j, t_{j+1}]$, i.e., $s^k_{t_{j+1}}/n_k = s^k_{t_{j}}/n_k$, then
\begin{itemize}
\item for $\nzu_{k,t_j} < \nzu_{k,t_{j+1}}$ we get $\ynu_{k,t_j} < \ynu_{k,t_{j+1}}$, and
\item for $\nzu_{k,t_j} > \nzu_{k,t_{j+1}}$ we get $\ynl_{k,t_j} < \ynl_{k,t_{j+1}}$.
\end{itemize}
Again, this follows from the weighted average property of $\yktn$;
for constant $\yktz$, the lower $\nktz$ the closer $\yktn$ is to $s^k_t/n_k$.
% one can think of the update between interval and precise value: the lower $n$, the 'faster' the interval will shrink!)\\
Therefore, we propose to take the same $\nktz$ bounds for all $t$, or, at most, to have it change with $t$ gradually only.

%***generally require $\yktzl > 0$ and $\yktzu < 1$ for all $t \not\in \big(\min(\vec{t}^k), \max(\vec{t}^k)\big)$
%to get proper posteriors***
Generally, one should choose $\yktzl > 0$ and $\yktzu < 1$ for all $t \not\in \big(\min(\vec{t}^k), \max(\vec{t}^k)\big)$.

***what to do in extreme tail of component distributions?
How sensitive is the method to choosing prior (lower, upper) mean near-zero (how near?) for $p^k_t$ at high $t$'s?***

***elicitation: we should also consider the case where $\yktz$ bounds are not given for all $t \in {\cal T}$,
as it will be too time-consuming to elicit $\yktz$ bounds for all $t$ in a dense grid.
(Calculating $\Rsys(t)$ for a too coarse grid only will waste information from data.)
Idea: ``fill up the $t$ grid at all $t$'' with least-committal bounds, i.e.: 
at all $t$ not elicited, take $\yktzu$ equal to last (in time sequence) elicited $\yktzu$,
and $\yktzl$ equal to next (in time sequence) elicited $\yktzl$.

***possible elicitation procedure: starting with $\yktz$ bounds for `middle' or `typical' $t$,
then $\yktz$ bounds for $t$ smaller than `middle' $t$ (or halfway between 0 and `middle' $t$?) and for $t$ larger than `middle' $t$,
refine further as possible.

\section{Computations \& Examples}

***code published in Louis' \textbf{R} package \texttt{ReliabilityTheory}
(which contains the function to calculate the survival signature)***


***calculate and show $\Rsys(t)$ plots for a range of examples (keep system layout fixed?):
informative prior set for some component types, different degree of imprecision,
noninformative prior set for other components (make clear that precise `noninformative' prior actually contains info), 
combine with prior-data conflict and non-conflict test data,
to show effect of precision of prior knowledge / amount of data / (degree of) prior-data conflict on $\Rsys(t)$.***

***Vacuous prior sets $[\yktzl, \yktzu] = (0,1)$ for some component types,
informative prior sets for others. $[\nktzl, \nktzu]$ interval for all component types.***

***Some extra illustrations? E.g. the effect of extra redundancy on $\Rsys(t)$ intervals like in Risk Analysis paper?
(could nicely illustrate challenges in decision making with intervals:
do I prefer adding redundancy such that there is a chance for a large effect, but high uncertainty for it,
or do I rather add redundancy such that effect is smaller but more certain?***

***Other interesting extra illustration:
show effect of replacement of components in system at certain time points
(could be corrective (component failed) or preventive replacement),
replaced components constitute a new type with shifted component reliability function***


\section{Conclusions and Outlook}

Upscaling the survival signature to large real-world systems and networks, consisting of thousands of components, is a major challenge.
However, even for such systems the fact that one only needs to derive the survival signature once for a system is an advantage,
and also the monotonicity of the survival signature for coherent systems is very useful if one can only derive it partially.  


***Extend the model to deal with right-censored observations which are common in the reliability setting.
We believe that a minimal assumption (component can fail immediately after censoring or live forever)
will be simple to implememt but will lead to high imprecision,
whereas assuming exchangeability with other surviving components at moment of censoring
will be more complex to accomodate but will lead to less imprecision.

***right-censoring will also allow to use data from running system,
to calculate its remaining useful life (RUL).



% ------------ bibliography -------------

\section*{References}

\bibliographystyle{elsarticle-harv}
\bibliography{ijar-npb-refs}

\end{document}

