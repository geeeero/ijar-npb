\documentclass[12pt, a4paper]{elsarticle}

% ------------ packages -------------

\usepackage[utf8]{inputenc}
\usepackage[OT1]{fontenc}
\usepackage{graphicx}
\usepackage[english]{babel}

\usepackage{amsmath}
\usepackage{amsfonts}
\usepackage{amssymb}
\usepackage{amsthm}

\usepackage[usenames,dvipsnames]{xcolor}
\usepackage{booktabs}
\usepackage{todonotes}
\usepackage{etoolbox}
\usepackage{url}
%\usepackage{tikz}
%\usetikzlibrary{shapes.misc,fit}

\usepackage[bookmarks]{hyperref}

% ------------ custom defs -------------

\newtheorem{theorem}{Theorem}
\newtheorem{lemma}[theorem]{Lemma}
\newtheorem{proposition}[theorem]{Proposition}
\newtheorem{corollary}[theorem]{Corollary}

\newcommand{\reals}{\mathbb{R}}
\newcommand{\posreals}{\reals_{>0}}
\newcommand{\posrealszero}{\reals_{\ge 0}}
\newcommand{\naturals}{\mathbb{N}}

\newcommand{\mbf}[1]{\mathbf{#1}}
\newcommand{\bs}[1]{\boldsymbol{#1}}
\renewcommand{\vec}[1]{{\bs#1}}

\newcommand{\uz}{^{(0)}} % upper zero
\newcommand{\un}{^{(n)}} % upper n
\newcommand{\ui}{^{(i)}} % upper i

\newcommand{\ul}[1]{\underline{#1}}
\newcommand{\ol}[1]{\overline{#1}}

\newcommand{\Rsys}{R_\text{sys}}
\newcommand{\lRsys}{\ul{R}_\text{sys}}
\newcommand{\uRsys}{\ol{R}_\text{sys}}

\newcommand{\Fsys}{F_\text{sys}}
\newcommand{\lFsys}{\ul{F}_\text{sys}}
\newcommand{\uFsys}{\ol{F}_\text{sys}}

\def\Tsys{T_\text{sys}}

\newcommand{\E}{\operatorname{E}}
\newcommand{\V}{\operatorname{Var}}

\newcommand{\indic}{\mathbb{I}}

\newcommand{\ber}{\operatorname{Bernoulli}} 
\newcommand{\bin}{\operatorname{Binomial}}
\newcommand{\be}{\operatorname{Beta}} 
\newcommand{\bebin}{\operatorname{Beta-Binomial}} 

\def\tmax{t_\text{max}}
\def\tnow{t_\text{now}}
\def\tpnow{t^+_\text{now}}

\newcommand{\ptk}{p^k_t}

\input{nydefs.tex}

\begin{document}

As example to illustrate the potential of our model,
we consider a simplified automotive brake system.
The master brake cylinder (M) activates all four wheel brake cylinders (C1 -- C4),
which in turn actuate a braking pad assembly each (P1 -- P4).
The hand brake mechanism (H) goes directly to the brake pad assemblies P3 and P4;
the car brakes when at least one brake pad assembly is actuated.
The system layout is depicted in Figure~\ref{fig:brakesystem},
together with prior and posterior sets of reliability functions for the four component types and the complete system.
Observed lifetimes from test data are indicated by tick marks in each of the four component type panels,
where $n_\text{M}=5$, $n_\text{H}=10$, $n_\text{C}=15$, and $n_\text{P}=20$.
We assume $[\ntzl{\text{M}},\ntzu{\text{M}}] = [1,8]\ \forall t$,
and $[\nktzl, \nktzu] = [1,2]$ for $k \in \{\text{H, C, P}\}$ and all $t$.
Prior functioning probability bounds for M are based on
a Weibull cdf with shape $2.5$ and scales $6$ and $8$ for the lower and upper bound, respectively.
The prior bounds for P can be seen as the least committal bounds
derived from an expert statement of $\ytz{\text{P}} \in [0.5, 0.65]$ for $t=5$ only.

***We see that posterior lower and upper functioning probabilities drop at those times $t$
for which there is a failure time in the test data,
or a drop in the prior functioning probability bounds.
The lower bound for prior system reliability drops to zero at $t=5$
since the prior lower bound for P drops to zero at $t=5$;
for the system to function, at least one brake pad assembly must function.***

For H, near-noninformative prior functioning probability bounds have been selected;
with the upper bound for P being approximately one for $t \le 5$ as well,
the prior upper system reliability bound for $t \le 5$ is close to one, too,
since the system can function on H and one of P1 -- P4 alone.
Note that the posterior functioning probability interval for M
is wide not only due to the limited number of observations,
but also because $\ntzu{\text{M}} = 8$ and the prior-data conflict reaction.

Posterior functioning probability bounds for the complete system
are much more precise than the prior system bounds,
reflecting the information gained from component test data.
The posterior system bounds can be also seen to reflect location and precision of the component bounds;
for example, the system bounds drop drastically between $t=2.5$ and $t=3.5$
mainly due to the drop of the bounds for P at that time.


\begin{figure}
\begin{tikzpicture}
[typeM/.style={rectangle,draw,fill=black!20,thick,inner sep=0pt,minimum size=5mm,font=\footnotesize},
 typeC/.style={rectangle,draw,fill=black!20,thick,inner sep=0pt,minimum size=5mm,font=\footnotesize},
 typeP/.style={rectangle,draw,fill=black!20,thick,inner sep=0pt,minimum size=5mm,font=\footnotesize},
 typeH/.style={rectangle,draw,fill=black!20,thick,inner sep=0pt,minimum size=5mm,font=\footnotesize},
 cross/.style={cross out,draw=red,very thick,minimum width=7mm, minimum height=5mm},
 hv path/.style={thick, to path={-| (\tikztotarget)}},
 vh path/.style={thick, to path={|- (\tikztotarget)}}]
\node at (0,0) {\includegraphics[width=\textwidth]{brakingsystem-2}};
\begin{scope}[scale=1.00,xshift=3.4cm,yshift=-1.1cm]
\node[typeM] (M)    at ( 0  , 0  ) {M};
\node[typeC] (C1)   at ( 1  , 1.5) {C1};
\node[typeC] (C2)   at ( 1  , 0.5) {C2};
\node[typeC] (C3)   at ( 1  ,-0.5) {C3};
\node[typeC] (C4)   at ( 1  ,-1.5) {C4};
\node[typeP] (P1)   at ( 2  , 1.5) {P1};
\node[typeP] (P2)   at ( 2  , 0.5) {P2};
\node[typeP] (P3)   at ( 2  ,-0.5) {P3};
\node[typeP] (P4)   at ( 2  ,-1.5) {P4};
\node[typeH] (H)    at ( 0  ,-1  ) {H};
\coordinate (start)  at (-0.7, 0);
\coordinate (startC) at ( 0.5, 0);
\coordinate (startH) at (-0.4, 0);
\coordinate (Hhop1)  at ( 0.4,-1);
\coordinate (Hhop2)  at ( 0.6,-1);
\coordinate (endP)   at ( 2.5, 0);
\coordinate (end)    at ( 2.8, 0);
\path (start)     edge[hv path] (M.west)
      (M.east)    edge[hv path] (startC)
      (startC)    edge[vh path] (C1.west)
                  edge[vh path] (C2.west)
                  edge[vh path] (C3.west)
                  edge[vh path] (C4.west)
      (C1.east)   edge[hv path] (P1.west)
      (C2.east)   edge[hv path] (P2.west)
      (C3.east)   edge[hv path] (P3.west)
      (C4.east)   edge[hv path] (P4.west)
      (endP)      edge[vh path] (P1.east)
                  edge[vh path] (P2.east)
                  edge[vh path] (P3.east)
                  edge[vh path] (P4.east)
                  edge[hv path] (end)
      (startH)    edge[vh path] (H.west)
      (H.east)    edge[hv path] (Hhop1)
      (Hhop1)     edge[thick,out=90,in=90] (Hhop2)
      (Hhop2)     edge[hv path] (P3.south)
                  edge[hv path] (P4.north);
\end{scope}
\end{tikzpicture}
\caption{Prior and posterior sets of reliability functions for a simplified automotive brake system
with layout as depicted in the lower right panel.}
\label{fig:brakesystem}
\end{figure}

\end{document}
